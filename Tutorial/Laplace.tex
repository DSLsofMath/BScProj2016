\documentclass{article}
\usepackage[utf8]{inputenc}

\title{Kapitel: Laplace Transform}
\author{ }
\date{March 2016}

\begin{document}

\maketitle

\section{Laplacetransform}

%TODO: general comment: spell checking

% TODO: general comment: use my comments to consider adding some
% explanation (here or earlier) to the readers about easy mistakes to
% make. (If you make mistakes, you can bet _many_ others will make the
% same, or similar, mistakes. So these mistakes can be seen as a
% resource when it comes to making a pedagogical end result.)

%Är osäker på hur mycket detaljer jag ska gå in i. This is a ROUGH draft. More like a template if anything else.
Laplacetransformen kan ses som ett specialfall av Fouriertransformen som bara är definerad för $t>0$
Dess definition är:
$$F = \mathcal{L} f = \hat{f}(-js) = \int_{0}^{\infty} f(t)e^{-st} dt $$

%där s är ett komplex tal.
där $s = \sigma + j \omega$ och båda är reella. Vi kan observera att om $\sigma = 0$ så har vi definitionen
för en Fouriertransform. Var nu inte rädda för att ni måste behandla komplexa tal, Laplacetransformen är
egentligen en mer "vänlig" transform jämfört med Fouriertransformen. 
%Och förmodligen kommer en inte märka av att det är ett komplex tal alls.
Och precis som Fouriertransformen så vill vi transformera tillbaka funktionerna. 
Den inversa Laplacetransformen är given ur:
%Invers Laplacetransform
$$f = \mathcal{L}^{-1} (F) = \frac{1}{2 \pi i} \int_{b-jr}^{b+jr} F(s) e^{st} ds $$ 
%Ska F(s) -> F inuti integralen med?
där f ska konvergera punktvis när $r\rightarrow \infty$.

%Vi kanske också borde förklara dess användning inom signalteori och grejer
Den mest användbara egenskapen av Laplace funktion är dess förmåga att hantera differentialekvationer.
%TODO: LHS uses a variable "t" but RHS uses (unbound) "s"
$$\mathcal{L} (f') = s F - f(0)$$
eller
%TODO: LHS uses a variable "t" but RHS uses (unbound) "s"
$$\mathcal{L} (f^{(k)}) = s^k F - \sum_{0}^{k-1} z^{k-1-j} f^{(j)} (0)$$

Den har även följande egenskap 
$$\mathcal{L} (t f) = -F' $$
%Andra viktiga transformer
%Frekvensförskjutning \mathcal{L} (f e^{ct}) = F(s-c)
%Tidsförskjutning \mathcal{L} (H(t-a) f(t-a)) = e^{-as) F(s)
%Tidsskalning \mathcal{L} (f) (at) = \frac{1}{a} F(z/a)
%Dirac Delta \mathcal{L} (\delta)(t) = 1 
%Ska dessa tas upp som exempel eller bara nämnas?

%TODO: Use "(" ")" to clarify what \mathcal{L} is applied to - Done
%TODO: Fix the variable binding
Ett vanligt fel man gör vid Laplacetransformering är att man tror att transformen av en produkt mellan två funktioner är produkten av varderas transform. 
Detta är oftast inte fallet! 
$$\mathcal{L} (f g) \neq L G $$
Observera att transformen av en faltning ger produkten av varderas transform!
$$\mathcal{L} (f*g) = F G $$ % Ska jag använda denna notering eller kodens discConv f g
\appendix
%Signals, System and Transforms från Charles L. Phillips, John M. Parr
%Fourier Analysis and Its Apllications by Gerald B. Folland


\end{document}
